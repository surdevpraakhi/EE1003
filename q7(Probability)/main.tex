\let\negmedspace\undefined
\let\negthickspace\undefined
\documentclass[journal]{IEEEtran}
\usepackage[a5paper, margin=10mm, onecolumn]{geometry}
%\usepackage{lmodern} % Uncomment if needed for pdflatex
\usepackage{tfrupee} % Include tfrupee package

\setlength{\headheight}{1cm} % Set the height of the header box
\setlength{\headsep}{0mm}     % Set the distance between the header box and the top of the text

\usepackage{gvv-book}
\usepackage{gvv}
\usepackage{cite}
\usepackage{amsmath,amssymb,amsfonts,amsthm}
\usepackage{algorithmic}
\usepackage{graphicx}
\usepackage{textcomp}
\usepackage{xcolor}
\usepackage{txfonts}
\usepackage{listings}
\usepackage{enumitem}
\usepackage{mathtools}
\usepackage{gensymb}
\usepackage{comment}
\usepackage[breaklinks=true]{hyperref}
\usepackage{tkz-euclide} 
\usepackage{listings}
%\usepackage{gvv}                                        
\def\inputGnumericTable{}                                 
\usepackage[latin1]{inputenc}                                
\usepackage{color}                                            
\usepackage{array}                                            
\usepackage{longtable}                                       
\usepackage{calc}                                             
\usepackage{multirow}                                         
\usepackage{hhline}                                           
\usepackage{ifthen}                                           
\usepackage{lscape}
\usepackage{tikz}
\usepackage{circuitikz}
\usepackage{standalone} % For including external TikZ files

\begin{document}

\bibliographystyle{IEEEtran}
\vspace{3cm}

\title{11.16.3.5.2}
\author{EE24BTECH11066 - YERRA AKHILESH}
% \maketitle
% \newpage
% \bigskip
{\let\newpage\relax\maketitle}

\renewcommand{\thefigure}{\theenumi}
\renewcommand{\thetable}{\theenumi}
\setlength{\intextsep}{10pt} % Space between text and floats

\numberwithin{equation}{enumi}
\numberwithin{figure}{enumi}
\renewcommand{\thetable}{\theenumi}

\textbf{Question}:\\ 

A fair coin with 1 marked on one face and 6 on the other and a fair die are both tossed. Find the probability that the sum of numbers that turn up is 12.\\

\textbf{Solution: }\\
\textbf{Textual solution: }\\
The probability of a given event $A$ (where $A$: Sum of the numbers is 12) is computed by considering the mutual independence of the coin and the die. Since the coin and die are independent events, we calculate the probability of each event separately and then multiply them.

The probability of the coin showing 6 is:
\begin{align}
P(\text{Coin} = 6) &= \frac{1}{2}
\end{align}
The probability of the die showing 6 is:
\begin{align}
P(\text{Die} = 6) &= \frac{1}{6}
\end{align}

Since the events are independent, the probability of both events happening together is:
\begin{align}
P(A) &= P(\text{Coin} = 6) \times P(\text{Die} = 6) \\
     &= \frac{1}{2} \times \frac{1}{6} = \frac{1}{12}.
\end{align}

\textbf{Computational Solution:}


To compute the probabilities for tossing a fair coin and rolling a six-sided die, we rely on the **Probability Mass Function (PMF)** and **Cumulative Distribution Function (CDF)** with additional utilization of the **Z-transform** to derive results analytically.

\subsection*{Definitions}
\subsubsection*{Probability Mass Function (PMF)}
The PMF, denoted as $P_{\vec{Z}}(z)$, represents the probability of obtaining a specific value $z$ in the sample space of $\vec{Z}$. For the experiment:
\begin{align}
\vec{Z} = \vec{X} + \vec{Y},
\end{align}
where:
\begin{align}
\vec{X} &\in \{1, 6\} \quad \text{and} \quad \vec{Y} \in \{1, 2, 3, 4, 5, 6\}.
\end{align}
The combined sample space is:
\begin{align}
S = \{(x, y) \mid x \in \{1, 6\}, y \in \{1, 2, 3, 4, 5, 6\}\},
\end{align}
with $|S| = 12$ total outcomes.

The PMF is then given by:
\begin{align}
    P_{\vec{Z}}(z) &=
    \begin{cases}
        \frac{1}{12}, & z \in \{2, 3, 4, 5, 6, 8, 9, 10, 11, 12\}, \\
        \frac{1}{6}, & z = 7, \\
        0, & \text{otherwise}.
    \end{cases}
\end{align}

For the sum $\vec{Z}$ to equal 7, there are exactly two favorable outcomes in the sample space:
\begin{align}
\{(1, 6), (6, 1)\}.
\end{align}
Thus, the probability for $P_{\vec{Z}}(7)$ is:
\begin{align}
P_{\vec{Z}}(7) &= \frac{\text{Favorable outcomes}}{\text{Total outcomes}} \\
     &= \frac{2}{12} = \frac{1}{6}.
\end{align}

For example, the value of $\vec{Z}$ is obtained by summing the outcomes of the coin and the die. Each specific value $z$ corresponds to a distinct event in $S$. 

The distribution can also be summarized as:
\begin{itemize}
    \item $P_{\vec{Z}}(2) = \frac{1}{12}$
    \item $P_{\vec{Z}}(3) = \frac{1}{12}$
    \item $\vdots$
    \item $P_{\vec{Z}}(7) = \frac{1}{6}$
    \item $\vdots$
    \item $P_{\vec{Z}}(12) = \frac{1}{12}$
\end{itemize}

\noindent where all $P_{\vec{Z}}(z)$ sum to 1, as expected:
\begin{align}
\sum_{z \in \{2, 3, \dots, 12\}} P_{\vec{Z}}(z) = 1.
\end{align}


\subsubsection*{Cumulative Distribution Function (CDF)}
The CDF, $F_{\vec{Z}}(z)$, represents the cumulative probability of $\vec{Z}$ up to a given value $z$:
\begin{align}
F_{\vec{Z}}(z) &= P(\vec{Z} \leq z) = \sum_{k=2}^{z} P_{\vec{Z}}(k),
\end{align}
where $z \in \{2, 3, \dots, 12\}$. Beyond $z = 12$, the CDF is 1:
\begin{align}
F_{\vec{Z}}(z) = 
\begin{cases} 
0, & z < 2, \\
\sum_{k=2}^{z} P_{\vec{Z}}(k), & 2 \leq z \leq 12, \\
1, & z > 12.
\end{cases}
\end{align}

\subsubsection*{Z-transform Representation}
The Z-transform provides an analytical tool to compute the PMF for discrete random variables. It is defined as:
\begin{align}
\mathcal{Z}[P_{\vec{Z}}(z)] = G_{\vec{Z}}(z) = \sum_{k=2}^{12} P_{\vec{Z}}(k) z^{-k},
\end{align}
where $G_{\vec{Z}}(z)$ is the generating function for $P_{\vec{Z}}(z)$.

\subsection*{Computational Process}

\subsubsection*{Step 1: PMF of $\vec{X}$ and $\vec{Y}$}
For the coin toss ($\vec{X}$):
\begin{align}
P_{\vec{X}}(x) &= 
\begin{cases} 
\frac{1}{2}, & x \in \{1, 6\}, \\
0, & \text{otherwise}.
\end{cases}
\end{align}

For the die roll ($\vec{Y}$):
\begin{align}
P_{\vec{Y}}(y) &= 
\begin{cases} 
\frac{1}{6}, & y \in \{1, 2, 3, 4, 5, 6\}, \\
0, & \text{otherwise}.
\end{cases}
\end{align}

\subsubsection*{Step 2: Z-transform of PMFs}
The Z-transform for $\vec{X}$ is:
\begin{align}
G_{\vec{X}}(z) &= \sum_{x \in \{1, 6\}} P_{\vec{X}}(x) z^{-x} \\
     &= \frac{1}{2} z^{-1} + \frac{1}{2} z^{-6}.
\end{align}

The Z-transform for $\vec{Y}$ is:
\begin{align}
G_{\vec{Y}}(z) &= \sum_{y=1}^{6} P_{\vec{Y}}(y) z^{-y} \\
     &= \frac{1}{6} (z^{-1} + z^{-2} + z^{-3} + z^{-4} + z^{-5} + z^{-6}).
\end{align}

\subsubsection*{Step 3: Z-transform of $\vec{Z} = \vec{X} + \vec{Y}$}
The PMF of $\vec{Z}$ is derived by multiplying the Z-transforms of $\vec{X}$ and $\vec{Y}$:
\begin{align}
G_{\vec{Z}}(z) &= G_{\vec{X}}(z) \cdot G_{\vec{Y}}(z).
\end{align}

Substituting $G_{\vec{X}}(z)$ and $G_{\vec{Y}}(z)$:
\begin{align}
G_{\vec{Z}}(z) &= \left(\frac{1}{2} z^{-1} + \frac{1}{2} z^{-6}\right) \cdot \frac{1}{6} (z^{-1} + z^{-2} + z^{-3} + z^{-4} + z^{-5} + z^{-6}).
\end{align}

Simplifying:
\begin{align}
G_{\vec{Z}}(z) &= \frac{1}{12} \left(z^{-2} + z^{-3} + z^{-4} + z^{-5} + z^{-6} + z^{-7} + z^{-7} + z^{-8} + z^{-9} + z^{-10} + z^{-11} + z^{-12}\right).
\end{align}

The coefficients of $z^{-k}$ in $G_{\vec{Z}}(z)$ give $P_{\vec{Z}}(k)$. For example:
\begin{align}
P_{\vec{Z}}(12) &= \text{Coefficient of } z^{-12} = \frac{1}{12}.
\end{align}

\subsubsection*{Step 4: CDF of $\vec{Z}$}
The CDF, $F_{\vec{Z}}(z)$, is obtained by summing the PMF values:
\begin{align}
F_{\vec{Z}}(z) &= \sum_{k=2}^{z} P_{\vec{Z}}(k).
\end{align}

For example:
\begin{align}
F_{\vec{Z}}(12) &= \sum_{k=2}^{12} P_{\vec{Z}}(k) = 1.
\end{align}

\subsection*{Conclusion}
Using the Z-transform, we analytically derive the PMF and compute the probability of obtaining a sum of $12$:
\begin{align}
P_{\vec{Z}}(12) &= \frac{1}{12}.
\end{align}

\textbf{Simulation:} \\
\begin{enumerate}
    \item Generate a random number for the coin toss using the \textbf{rand()} function.
    \item Restrict the random number to either $1$ or $6$ by using the \textbf{rand()} \% 2 operator and then assign it to the variable \textbf{coin}, where:
    \begin{itemize}
        \item If the random number is $0$, assign \textbf{coin} = 1.
        \item If the random number is $1$, assign \textbf{coin} = 6.
    \end{itemize}
    \item Generate a random number for the die roll using the \textbf{rand()} function.
    \item Restrict the random number to the range from 1 to 6 by using \textbf{rand()} \% 6 + 1, and assign it to the variable \textbf{die}.
    \item Now, calculate the sum of the \textbf{coin} and \textbf{die} (i.e., $\vec{Z} = \vec{X} + \vec{Y}$).
    \item Count the number of times the sum $\vec{Z}$ equals a specific value (for example, 12) by iterating this process for a large number of trials.
    \item Divide the number of favourable outcomes (where the sum equals the target value) by the total number of trials to get the desired PMF for that sum.
    \item The CDF can then be simulated by summing the required PMFs.
\end{enumerate}



\begin{figure}[h]
\centering
\includegraphics[width=\columnwidth]{figs/pmf.png}
\caption{Probability Mass Funtion}
\label{fig:Plot1} 
\end{figure}

\begin{figure}[h]
\centering
\includegraphics[width=\columnwidth]{figs/cdf.png}
\caption{Cumulative Distribution Function}
\label{fig:Plot1} 
\end{figure}
\end{document}
